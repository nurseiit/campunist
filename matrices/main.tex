\documentclass{beamer}
\usepackage{listings}
\usepackage{xcolor}

\lstset {
    language=Python,
    backgroundcolor=\color{black!5}, % set background color
    basicstyle=\footnotesize, % basic font setting
}
% Theme of presentation
\mode<presentation>
{
	\usetheme{Madrid}
	\usecolortheme{orchid}
}

% Information about presentation
\title{Fast Matrix Exponentiation}
\subtitle{With Applications}
\institute{UNIST}
\author{Nurseiit Abdimomyn}

% Uncomment this, if you want the table of contents to pop up at
% the beginning of each section/subsection:
\AtBeginSection[]
{
  \begin{frame}<beamer>{Outline}
   \tableofcontents[currentsection,currentsubsection]
  \end{frame}
}

\iffalse
  \AtBeginSubsection[]
  {
    \begin{frame}<beamer>{Outline}
      \tableofcontents[currentsection,currentsubsection]
    \end{frame}
  }
\fi
\begin{document}

% Title slide
\begin{frame}
	\titlepage
\end{frame}

\iffalse
    \begin{frame}{Table of contents}
      \tableofcontents
    \end{frame}

    \begin{frame}{Introduction}
    	\begin{block}{\LaTeX\ Beamer}
    		LaTeX Beamer is a powerful tool to create presentations
    	\end{block}
    	\pause
    	\begin{block}{Export to PDF}
    		It produces platform-independent presentations in PDF
    	\end{block}	
    \end{frame}
\fi

\section{Introduction}

\iffalse
    \begin{frame}{Blocks}
    	\begin{block}{This is block}
    		Using \textit{blocks} is a handy way to present your text
    	\end{block}
    	\pause
    	\begin{block}{Pause between}
    		You can also insert a pause between blocks
    	\end{block}	
    \end{frame}
\fi

\begin{frame}{Matrix Multiplication}
	Consider two matrices:
	\pause
	\begin{itemize}
		\item Matrix $A$ is $n * k$ dimensional.
		\pause
		\item Matrix $B$ is $k * m$ dimensional.
	\end{itemize}
	\pause
	Notice that $A$'s columns and $B$'s rows number are identical!
	\pause \newline \newline
	Then we define matrix $C = A * B$ as:
	\newline
	 \[
	 \left[ {\begin{array}{cccc}
       c_{11} & c_{12} & ... & c_{1m} \\
       c_{21} & c_{22} & ... & c_{2m} \\
         ...   &   ...   & ... &   ...   \\
       c_{n1} & c_{n2} & ... & c_{nm}
      \end{array} } \right]
      =
      \left[ {\begin{array}{cccc}
       a_{11} & a_{12} & ... & a_{1k} \\
       a_{21} & a_{22} & ... & a_{2k} \\
         ...   &   ...   & ... &   ...   \\
       a_{n1} & a_{n2} & ... & a_{nk}
      \end{array} } \right]
      *
      \left[ {\begin{array}{cccc}
       b_{11} & b_{12} & ... & b_{1m} \\
       b_{21} & b_{22} & ... & b_{2m} \\
         ...   &   ...   & ... &   ...   \\
       b_{k1} & b_{k2} & ... & b_{km}
      \end{array} } \right]
    \]
    \pause \newline
    Thus, $C$ is an $n * m$ dimensional matrix.
    \newline 
    Which is calculated as: 
    $c_{ij} = \sum\limits_{r=1}^k a_{ir} * b_{rj}$
\end{frame}

\begin{frame}{Couple of things to notice}
	\begin{itemize}
	    \item 
    	    Matrix $C$ has $n * m$ elements and each element is computed in $k$ steps with given formula.
    	    \pause \newline
    	    Thus, we can obtain $C$ in $O(n * m * k)$, given $A$ and $B$.
    	\pause
	    \item
	        If $n = m = k$ (i.e. both $A$ and $B$ have $n$ rows and $n$ columns), 
	        \newline 
	        then $C$ has $n$ rows and $n$ columns, and can be computed in $O(n^3)$.
	\end{itemize}
\end{frame}

\begin{frame}{Some useful properties of matrix multiplication}
	\begin{itemize}
	    \item 
    	    It is $not$ $commutative$: $A * B \neq B * A$ in general case.
    	\pause
	    \item
	        It is $associative$: \newline $A * B * C = (A * B) * C = A * (B * C)$ in case $A * B * C$ exists;
        \pause	    
            \item
	        If you have a matrix with $n$ rows and $n$ columns,
	        \newline then multiplying it by $I_n$ gives the same matrix. 
	        \newline i. e. $I_n * A = A * I_n = A$.
	        \pause \newline Where $I_n$ is a matrix with $n$ rows and $n$ columns of such form:
	        \[
	          I_n = 
        	  \left[ {\begin{array}{cccc}
                    1 & 0 & ... & 0 \\
                    0 & 1 & ... & 0 \\
                    ...   &   ...   & ... &   ...   \\
                    0 & 0 & ... & 1
                    \end{array} } 
                  \right]
                \]
	\end{itemize}
\end{frame}

\section{Matrix exponentiation}

\begin{frame}[fragile]{Matrix exponentiation}
    Suppose we need to find some $n*n$ dimensional square matrix $A$ to the power $p$ or $A^p$
    \pause \newline We can do so via:
      \begin{lstlisting}
      function matpow_naive(A, p):
        result = I_n
        for i = 1..p:
          result = result * A
        return result
      \end{lstlisting}
    Which will run in $O(n^3 * p)$
\end{frame}

\begin{frame}[fragile]{Fast Matrix exponentiation}
    Can we do it any faster?
    \pause \newline Yes, we can apply the $BinPower$ algorithm here:
      \begin{lstlisting}
      function matBinPow(A, p):
        result = I_n
        while p > 0:
          if p % 2 == 1:
            result = result * A
          A = A * A
          p = p / 2
        return result
      \end{lstlisting}
    Which will run in $O(n^3 * \log{p})$
\end{frame}

\section{Applications}

\begin{frame}{Finding Nth Fibonacci number}
  Fibonacci numbers, $F_n$ are defined as:
  \begin{itemize}
    \item $F_0 = F_1 = 1$
    \item $F_i = F_{i-1} + F_{i-2}$ for $i > 1$.
  \end{itemize}
  \pause
  We want to calculate $F_n$ $mod$ $M$, where $n < 10^{18}$ and $M = 10^9 + 7$.
\end{frame}

\begin{frame}{Finding Nth Fibonacci number}
  Suppose we have a vector (matrix with 
  one row and several columns) of $(F_{i-2}, F_{i-1})$
  and we want to multiply it by some matrix, so that we get $(F_{i-1}, F_i)$ as a result.
  \pause \newline
  Let's call this matrix $M$: \newline
  $(F_{i-2}, F_{i-1}) * M = (F_{i-1}, F_i)$ \newline \newline
  Two questions we should answer arise immediately:
  \begin{enumerate}
    \item What are the dimensions of $M$?
    \item What are the exact values in $M$?
  \end{enumerate}
\end{frame}

\begin{frame}{Finding Nth Fibonacci number}
  We can answer them using the definition of matrix multiplication: \newline
  \pause
  \begin{itemize}
    \item The $size$. \newline
      We multiply $(F_{i-2}, F_{i-1})$, which has $1$ row and $2$ columns, by $M$. 
      \newline
      The result is $(F_{i-1}, F_i)$, which has $1$ row and $2$ columns.
      \pause \newline \newline
      By definition, if we multiply a matrix with $N$ rows and $K$ columns by 
      a matrix with $K$ rows and $L$ columns, we get a matrix with $N$ rows and $L$ columns.
      \newline \newline
      Therefore, matrix $M$ has $K = 2$ rows and $L = 2$ columns.
  \end{itemize}
\end{frame}

\begin{frame}{Finding Nth Fibonacci number}
  We can answer them using the definition of matrix multiplication: \newline
  \begin{itemize}
    \item The $values$. \newline \pause
      We now know that $M$ has $2$ rows and $2$ columns, $4$ values overall.
      Let’s denote them by letters, as we usually do with unknown variables:
      \newline \newline
        \[
          M = 
          \left[ {
            \begin{array}{cc}
              a & b \\
              c & d \\      
            \end{array} } 
          \right]
        \]
      \pause
      We want to find $a, b, c$ and $d$.
      \newline
  \end{itemize}
\end{frame}

\begin{frame}{Finding Nth Fibonacci number}
  We have, 
  \[
    M = 
    \left[ {
      \begin{array}{cc}
        a & b \\
        c & d \\      
      \end{array} } 
    \right]
  \]
  \pause
  We want to find $a, b, c$ and $d$.
  \newline
  Let's see what we get if we multiply $(F_{i-2}, F_{i-1})$ by $M$ by definition:  
  \newline \pause
  \[
    (F_{i-2}, F_{i-1})
    *
    \left[ {
      \begin{array}{cc}
        a & b \\
        c & d \\      
      \end{array} } 
    \right]
    =
    (a * F_{i-2} + c * F_{i-1}, b * F_{i-2} + d * F_{i-1})
  \]
  \pause
  Moreover, we know that the result of this multiplication must be $(F_{i-1}, F_i)$:
  \newline
  $(a * F_{i-2} + c * F_{i-1}, b * F_{i-2} + d * F_{i-1}) = (F_{i-1}, F_i)$
\end{frame}

\begin{frame}{Finding Nth Fibonacci number}
  Now we can write the system of equations:
  \begin{itemize}
    \item 
      $a * F_{i-2} + c * F_{i-1} = F_{i-1}$
    \item
      $b * F_{i-2} + d * F_{i-1} = F_i$
  \end{itemize}
  \pause
  The easiest solution is to set:
  \newline
  $a = 0, b = 1, c = 1, d = 1$
  \pause \newline
  Thus, we obtain:
  \[
    (F_{i-2}, F_{i-1})
    *
    \left[ {
      \begin{array}{cc}
        0 & 1 \\
        1 & 1 \\      
      \end{array} } 
    \right]
    = (F_{i-1}, F_i)
  \]
\end{frame}

\begin{frame}{Finding Nth Fibonacci number}
  You may wonder why on Earth would we need to overcomplicate the problem, introduce matrices or else?
  \newline Hang on, you'll see in a moment.
  \pause \newline
  Initially, we have $F_0$ and $F_1$.
  \newline
  $(F_0, F_1) = (1, 1)$
  \pause \newline
  By multiplying this vector with the matrix M we get:
  \newline
  $(F_1, F_2) = (1, 2)$ or $(1, 1) * M = (1, 2)$.
  \pause \newline
  If we multiply $(1, 2)$ by $M$:
  \newline
  $(1, 2) * M = (2, 3)$
  \pause \newline
  But we could get the same result by multiplying $(1, 1)$ by $M^2$.
  \newline
  $(1, 1) * M^2 = (2, 3)$.
  \pause \newline Now, can you see the pattern?
\end{frame}

\begin{frame}{Finding Nth Fibonacci number}
  In general, we get:
  \newline
  $(1, 1) * M^k = (F_k, F_{k+1})$
  \pause \newline
  Computing $M^k$ takes $O(2^3 * log(k))$ time. 
  \newline
  Thus, we can now find $N$-th Fibonacci number in $O(log(N))$ time.
\end{frame}

\begin{frame}{General Solution for Recurrence Relations}
  
\end{frame}

\begin{frame}{}
		\LARGE{\centerline{Thank you!}}
\end{frame}

\end{document}
